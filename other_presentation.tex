\documentclass{beamer}
 
\usepackage[utf8]{inputenc}
\usepackage{hyperref}
%\usepackage{verbatim}
\usepackage{listings}
 
%Information to be included in the title page:
\title{CLTK}
\author{Eleftheria, Clément}
\institute{CLTK}
\date{06/10/2019}


%\lstset{language=TeX}

\AtBeginSection[]
{
\begin{frame}{Plan}
\tableofcontents[currentsection]
\end{frame}
}

\AtBeginSubsection[]
{
\begin{frame}{Plan}
\tableofcontents[currentsubsection]
\end{frame}
}

\begin{document}

\begin{frame}
%\titlepage
\begin{center}
    {\large Introduction to CLTK}\\
    Eleftheria Chatziargyriou \& Clément Besnier \\
    06/11/2019 \\
    \includegraphics[scale=0.5]{cltklogo.png}
\end{center}

\end{frame}

\begin{frame}
\frametitle{Outline}
\tableofcontents
\end{frame}

\section{CLTK: philosophy and organization}

% \begin{frame}
% \frametitle{Brief overview on CLTK}
% %like page 5
% \begin{itemize}
%     \item definition
%     \item license
%     \item started in 2012
%     \item covered languages
%     \item number of citations
% \end{itemize}
% \end{frame}

\begin{frame}{CLTK}
\begin{itemize}
    \item \underline{Free and Open-Source} \underline{Python} library
    \pause
    \item Founded in 2012 by \underline{Kyle P. Johnson}
    \pause
    \item Provides \underline{NLP tools} for \underline{historical languages}
    \pause
    \item \underline{Shares} a \underline{high-quality code} for \underline{academic research}
    
\end{itemize}{}
    
\end{frame}{}

\subsection{NLP Tools}
\begin{frame}
\frametitle{CLTK among other NLP Tools in Python}
\begin{itemize}
    
    \item SpaCy\footnote{\href{https://spacy.io/}{https://spacy.io/}}, NLTK\footnote{\href{https://www.nltk.org/}{https://www.nltk.org/}}, StanfordNLP\footnote{\href{https://stanfordnlp.github.io/stanfordnlp/}{https://stanfordnlp.github.io/stanfordnlp/}}
    \item Python is a programmaing language widely used by researchers
\end{itemize}
\end{frame}


\subsection{Historical Languages}

\begin{frame}
\frametitle{Historical Languages}
\begin{itemize}
    \item Early Antiquity: Sumerian, Akkadian, Old Egyptian, etc
    \item Late Antiquity: Ancient Greek, Latin, Sanskrit, Classical Chinese, Gothic, Proto-Nordic, etc
    \item Middle Ages: Medieval Latin, Coranic Arabic, Koine, Old and Middle High German, Old Norse, etc
\end{itemize}
\end{frame}

\begin{frame}
\frametitle{Historical Languages}
\begin{itemize}
    \item Handles languages that let traces before the Gutenberg's printing  invention
    \item Documents written in these languages have specific features:   
    \begin{itemize}
        \item few and likely indirect remaining traces
        \item spelling not normalized
        \item dialect boundaries not clearly observable
        %\item language evolutions must be taken into account
        \item unavoidable language dynamics  
        \item no more living speakers
        \item no more produced texts
        
    \end{itemize}
    \item Expert skills needed
\end{itemize}
\end{frame}


\subsection{High Quality Code}

\begin{frame}
\frametitle{Principles}
\begin{itemize}
    \item Decentralization
    \item Disintermediation
    \item Extensibility
    \item Standardization
    \item Simplicity
\end{itemize}{}
    
\end{frame}


\begin{frame}
\frametitle{Community Design Principles}
\begin{itemize}
    \item Free \& Open Source
    \item Transparency
    \item Inclusion
    \item Multi-disciplinary
    \item Mutual benefit
\end{itemize}
\end{frame}


\begin{frame}
\frametitle{Free, open-source, decentralized}
\begin{itemize}
    \item Free  %
    \item MIT license\footnote{\href{https://choosealicense.com/licenses/mit/}{https://choosealicense.com/licenses/mit/}}, you can share and reuse it, even for commercial code
    \item Inclusion % researchers, like students and passionate people
    \item Multi-disciplinary % computer science, linguistics, statistics, history
    %\item Mutual benefit % consequence of the previous remark
\end{itemize}
\end{frame}



\begin{frame}[fragile]
\frametitle{Academic research}
    \begin{lstlisting}[basicstyle=\footnotesize]
    @Misc{johnson2014,
        author = {Kyle P. Johnson et al.},
        title = {CLTK: The Classical Language Toolkit},
        howpublished = {\url{https://github.com/cltk/cltk}},
        note = {{DOI} 10.5281/zenodo.<current_release_id>},
        year = {2014--2019},
    }
    \end{lstlisting}
%\begin{itemize}
%\end{itemize}
\end{frame}

\section{CLTK - Code and Contribution}

%\begin{frame}
%\frametitle{Goals}
%\begin{itemize}
%    \item Compile analysis-friendly corpora
%    \item Collect and generate linguistic data
%\end{itemize}
%\end{frame}


\subsection{Code}

\begin{frame}
\frametitle{What can CLTK do?}
\begin{itemize}
    \item Corpora importing
    \item Text preprocessing
    \begin{itemize}
        \item File Parsing
        \item Orthographic Normalization
        \item ASCII/Unicode Conversion
        \item Stopword Filtering
        \item Syllabification
        \item Syllable/Word Stressing
        \item Phonetic Indexing
        \item Word/line Tokenization
        \item IPA Transcription
        \item Lemmatization
        \item Stemming
        \item POS Tagging
        \item Poetry Scansion
        \item Named Entity Recognition
    \end{itemize}
    % Add a schema where we can show some interesting pipelines -> examples for scansion, semantic analysis, etc
\end{itemize}
\end{frame}



\begin{frame}
\frametitle{Currently Supported Languages}
Akkadian, Arabic, Bengali, Chinese, Coptic, Ancient Egyptian, Old English, Middle English, French,
Middle High German, Middle Low German, Gothic, Greek, Gurajati, Hebrew, Hindi, Javanese,
Kannada, Latin, Malayalam, Marathi, Old Norse, Odia, Ottoman, Pali, Persian, Old Portuguese,
Prakrit, Punjabi, Sanskrit, Old Swedish, Tami, Telugu, Tocharian B, Urdu
\end{frame}

% Add a table with languages and their tools


\subsection{Contribution}

\begin{frame}
\frametitle{Contribution}
\begin{itemize}
    \item Collaborative effort / open to a virtually infinite talent pool
    \item Avoid “re-inventing the wheel”
    \item Closer to the needs of the community
    \item Constant patches 
    \begin{itemize}
    \item Bugs are quickly resolved 
    \item New features are constantly developed)
    \item Transparency of development
    \item Generally results in safer software
    \item Easily customizable
    \end{itemize}
\end{itemize}
\end{frame}

\begin{frame}
\frametitle{Why contribute?}
\begin{itemize}
    \item Expand your skill set
    \item Give back to the community
    \item Open Source culture
    \item It’s Fun!
\end{itemize}
\end{frame}




\begin{frame}
\frametitle{How to contribute}
\begin{itemize}
    \item You can check out the CLTK tutorials (https://github.com/cltk/tutorials) and docs (http://docs.cltk.org)
    \item Take a look at the open issues (https://github.com/cltk/cltk/issues) or simply make your own contribution.
    \item Don’t hesitate to ask for help in the IRC channel (https://gitter.im/cltk/cltk)!
\end{itemize}
\end{frame}




\begin{frame}
\frametitle{Summary}
\begin{itemize}
    \item Digital tools can be used to aid academics and speed up mundane and well-defined processes
    \item Classical languages have their own unique set of challenges compared to modern languages
    \item CLTK offers an easy to use and well-documented API for Classical Natural Language Processing
\end{itemize}
\end{frame}


\begin{frame}
\begin{center}
    Thank you for your attention!
\end{center}


\end{frame}




% \begin{frame}
% \frametitle{Open source}
% \href{http://github.com}{github.com}
% Screenshot of the github code page.
% %like page 6

% \end{frame}

% \begin{frame}
% \frametitle{The docs}

% Screenshot of the docs

% \end{frame}


% \begin{frame}
% \frametitle{Project}


% Even if no lab is supporting it.

% Similar projects exist for living languages: SpaCy, NLTK, StanfordNLP, etc. Add links here 
% \end{frame}
% \section{CLTK as a tool for digital classics}

% \begin{frame}
% \frametitle{Gathers corpora from different sources}
% \begin{itemize}
%     \item annotated corpora
%     \item raw corpora
% \end{itemize}
% Some standardization
% \begin{itemize}
%     \item common encoding: UTF-8
%     \item 
% \end{itemize}


% CLTK comes after OCR\footnote{Optical Character Recognition} task.
% \end{frame}

% \begin{frame}
% \frametitle{Processes texts}
% \begin{itemize}
%     \item sentence segmentation
%     \item tokenization
%     \item POS tagging
%     \item lemmatization
    
% \end{itemize}
% \end{frame}

% \begin{frame}
% \frametitle{Ensuring quality}

% \begin{itemize}
%     \item reproducibility
%     \item collaborations
%     \item availability for all
    
% \end{itemize}

% \end{frame}

% \section{Example}
% \begin{frame}
% \frametitle{}


% \end{frame}

% \begin{frame}
% \frametitle{}


% \end{frame}


% \begin{frame}
% \frametitle{}


% \end{frame}


% \begin{frame}
% \frametitle{}


% \end{frame}


% \begin{frame}
% \frametitle{}


% \end{frame}


% \begin{frame}
% \frametitle{}


% \end{frame}




 
\end{document}