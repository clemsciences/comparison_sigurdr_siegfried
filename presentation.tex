\documentclass{beamer}
 
\usepackage[utf8]{inputenc}
\usepackage{hyperref}
 
%Information to be included in the title page:
\title{CLTK}
\author{Eleftheria, Clément}
\institute{CLTK}
\date{06/10/2019}
 
 
 
\begin{document}
 
\begin{frame}
\titlepage
\end{frame}

\begin{frame}
\frametitle{Outline}
\tableofcontents
\end{frame}

\section{CLTK Presentation}

\begin{frame}
\frametitle{Brief overview on CLTK}
%like page 5
\begin{itemize}
    \item definition
    \item license
    \item started in 2012
    \item covered languages
    \item number of citations
\end{itemize}


\end{frame}

\begin{frame}
\frametitle{Open source}
\href{http://github.com}{github.com}
Screenshot of the github code page.
%like page 6

\end{frame}

\begin{frame}
\frametitle{The docs}

Screenshot of the docs

\end{frame}


\begin{frame}
\frametitle{Project}


Even if no lab is supporting it.

Similar projects exist for living languages: SpaCy, NLTK, StanfordNLP, etc. Add links here 
\end{frame}
\section{CLTK as a tool for digital classics}

\begin{frame}
\frametitle{Gathers corpora from different sources}
\begin{itemize}
    \item annotated corpora
    \item raw corpora
\end{itemize}
Some standardization
\begin{itemize}
    \item common encoding: UTF-8
    \item 
\end{itemize}


CLTK comes after OCR\footnote{Optical Character Recognition} task.
\end{frame}

\begin{frame}
\frametitle{Processes texts}
\begin{itemize}
    \item sentence segmentation
    \item tokenization
    \item POS tagging
    \item lemmatization
    
\end{itemize}
\end{frame}

\begin{frame}
\frametitle{Ensuring quality}

\begin{itemize}
    \item reproducibility
    \item collaborations
    \item availability for all
    
\end{itemize}

\end{frame}

\section{Example}
\begin{frame}
\frametitle{}


\end{frame}

\begin{frame}
\frametitle{}


\end{frame}


\begin{frame}
\frametitle{}


\end{frame}


\begin{frame}
\frametitle{}


\end{frame}


\begin{frame}
\frametitle{}


\end{frame}


\begin{frame}
\frametitle{}


\end{frame}




 
\end{document}