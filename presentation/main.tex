\documentclass{article}
\usepackage[T1]{fontenc}
\usepackage[utf8]{inputenc}
\usepackage{hyperref}

\title{Comparison between Siegfried and Sigur{\dh}r}

\author{Clément Besnier}
\date{2019}

\begin{document}

\maketitle

\section{Introduction}

\section{Resources}

\subsection{Nibelungenlied}

Nibelungenlied, \textit{the song of Nibelung}, was probably written in the XII century. 
A diplomatic version is available \cite{nibdiplo} and a normalized version  \cite{nibnor}. The language in which it was composed is in the Middle-High-German language.

We used the Referenzkorpus Mittelhochdeutsch \cite{refkormhd}. It is available under the Attribution-ShareAlike 4.0 International\footnote{\href{https://creativecommons.org/licenses/by-sa/4.0/}{https://creativecommons.org/licenses/by-sa/4.0/}} license. This corpus' main aim is to provide enough linguistic data for any automatic process on Middle High German \cite{dipper2015annotierte}.


The annotators provide useful and precise hints for each text. For \textit{Nibelungenlied}, they are: 
\begin{itemize}
    \item Text domain: poetry
    \item text kind: V
    \item date: ca. 1180-1210
    \item language: ostbairisch, österreichisch
    \item Edition:Ursula Hennig (Hg.), Das Nibelungenlied nach der Handschrift C (Altdeutsche Textbibliothek 83), Tübingen 1977
\end{itemize}


\subsection{Völsunga saga}

\cite{volsungasaga}


\nocite{*}
\bibliographystyle{alpha}
\bibliography{main}

\end{document}
