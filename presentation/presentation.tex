\documentclass{beamer}
  
\usepackage[utf8]{inputenc}
\usepackage{hyperref}
 
%Information to be included in the title page:
\title{CLTK}
\author{Eleftheria, Clément}
\institute{CLTK}
\date{06/10/2019}



\begin{document}

\begin{frame}
%\titlepage
\begin{center}
    {\large Introduction to CLTK}\\
    Eleftheria Chatziargyriou \& Clément Besnier \\
    06/11/2019 \\
    \includegraphics[scale=0.5]{cltklogo.png}
\end{center}

\end{frame}

\begin{frame}
\frametitle{Outline}
\tableofcontents
\end{frame}

\section{An introduction to CLTK}

% \begin{frame}
% \frametitle{Brief overview on CLTK}
% %like page 5
% \begin{itemize}
%     \item definition
%     \item license
%     \item started in 2012
%     \item covered languages
%     \item number of citations
% \end{itemize}
% \end{frame}

\section{Digital + Classics}
\begin{frame}{Why introduce digital tools to Humanities}
\begin{itemize}
    \item Speeds up the text cleaning and preprocessing stage
    \item Renders data aggregation/visualization easier
    \item Promotes interdisciplinarity
    \item Improves shareability
\end{itemize}{}
    
\end{frame}




\begin{frame}{Challenges with Digital Classics}

\begin{itemize}
    \item Resource Scarcity / few tagged corpora
    \item Licensing problems
    \item Lack of canonical spelling rules
    \item No “standard” dialect for most languages
\end{itemize}{}
    
\end{frame}{}



\begin{frame}
\frametitle{Challenges with Digital Classics}
\begin{itemize}
    \item Breadth vs Depth: \textit{Have models that are applicable to many dialects with lower accuracy, or accurate yet extremely   specific models?}

    \item Efficiency vs Readability: \textit{Have an easily readable and extendable codebase or focus on algorithmic efficiency?}
    \item 
\end{itemize}
\end{frame}


\section{CLTK: Philosophy and history}


\begin{frame}
\frametitle{What is CLTK?}
\begin{itemize}
    \item Open-Source Python library
    \item Founded in 2012 by Kyle P. Johnson
    \item Academic Advisors:
    \begin{itemize}
    \item Gregory Crane (Leipzig/Tufts)
    \item Neil Coffee (Buffalo)
    \item Peter Meineck (NYU)
    \item Leonard Muellner (Brandeis/CHS)
    \end{itemize}
\end{itemize}
\end{frame}




\begin{frame}
\frametitle{What is CLTK?}
\begin{itemize}
    \item Offers NLP support for Classical Languages
    \item Attempts to aid academic research
    \item Common API across all languages
\end{itemize}
\end{frame}





\begin{frame}
\frametitle{Coding Design Principles}
\begin{itemize}
    \item Decentralization
    \item Disintermediation
    \item Extensibility
    \item Standardization
    \item Simplicity
\end{itemize}
\end{frame}





\begin{frame}
\frametitle{Community Design Principles}
\begin{itemize}
    \item Free & Open Source
    \item Transparency
    \item Inclusion
    \item Multi-disciplinary
    \item Mutual benefit
\end{itemize}
\end{frame}




\begin{frame}
\frametitle{Goals}
\begin{itemize}
    \item Compile analysis-friendly corpora
    \item Collect and generate linguistic data
    \item Act as a free and open platform for generating scientific research
\end{itemize}
\end{frame}


\section{CLTK Features}



\begin{frame}
\frametitle{Currently Supported Languages}
Akkadian, Arabic, Bengali, Chinese, Coptic, Ancient Egyptian, Old English, Middle English, French,
Middle High German, Middle Low German, Gothic, Greek, Gurajati, Hebrew, Hindi, Javanese,
Kannada, Latin, Malayalam, Marathi, Old Norse, Odia, Ottoman, Pali, Persian, Old Portuguese,
Prakrit, Punjabi, Sanskrit, Old Swedish, Tami, Telugu, Tocharian B, Urdu
\end{frame}



\begin{frame}
\frametitle{What can CLTK do?}
\begin{itemize}
    \item Corpora
    \item Text preprocessing
    \begin{itemize}
        \item File Parsing
        \item Orthographic Normalization
        \item ASCII/Unicode Conversion
        \item Stopword Filtering
        \item Syllabification
        \item Syllable/Word Stressing
        \item Phonetic Indexing
        \item Word/line Tokenization
        \item IPA Transcription
        \item Lemmatization
        \item Stemming
        \item POS Tagging
        \item Poetry Scansion
        \item Named Entity Recognition
    \end{itemize}
    
\end{itemize}
\end{frame}



\begin{frame}
\frametitle{Why Open Source?}
\begin{itemize}
    \item Collaborative effort / open to a virtually infinite talent pool
    \item Avoid “re-inventing the wheel”
    \item Closer to the needs of the community
    \item Constant patches 
    \begin{itemize}
    \item Bugs are quickly resolved 
    \item New features are constantly developed)
    \item Transparency of development
    \item Generally results in safer software
    \item Easily customizable
    \end{itemize}
\end{itemize}
\end{frame}




\begin{frame}
\frametitle{Why contribute?}
\begin{itemize}
    \item Expand your skill set
    \item Give back to the community
    \item Open Source culture
    \item It’s Fun!
\end{itemize}
\end{frame}



\begin{frame}
\frametitle{How to contribute}
\begin{itemize}
    \item You can check out the CLTK tutorials (https://github.com/cltk/tutorials) and docs (http://docs.cltk.org)
    \item Take a look at the open issues (https://github.com/cltk/cltk/issues) or simply make your own contribution.
    \item Don’t hesitate to ask for help in the IRC channel (https://gitter.im/cltk/cltk)!
\end{itemize}
\end{frame}




\begin{frame}
\frametitle{Summary}
\begin{itemize}
    \item Digital tools can be used to aid academics and speed up mundane and well-defined processes
    \item Classical languages have their own unique set of challenges compared to modern languages
    \item CLTK offers an easy to use and well-documented API for Classical Natural Language Processing
\end{itemize}
\end{frame}


\begin{frame}
\begin{center}
    Thank you for your attention!
\end{center}


\end{frame}




% \begin{frame}
% \frametitle{Open source}
% \href{http://github.com}{github.com}
% Screenshot of the github code page.
% %like page 6

% \end{frame}

% \begin{frame}
% \frametitle{The docs}

% Screenshot of the docs

% \end{frame}


% \begin{frame}
% \frametitle{Project}


% Even if no lab is supporting it.

% Similar projects exist for living languages: SpaCy, NLTK, StanfordNLP, etc. Add links here 
% \end{frame}
% \section{CLTK as a tool for digital classics}

% \begin{frame}
% \frametitle{Gathers corpora from different sources}
% \begin{itemize}
%     \item annotated corpora
%     \item raw corpora
% \end{itemize}
% Some standardization
% \begin{itemize}
%     \item common encoding: UTF-8
%     \item 
% \end{itemize}


% CLTK comes after OCR\footnote{Optical Character Recognition} task.
% \end{frame}

% \begin{frame}
% \frametitle{Processes texts}
% \begin{itemize}
%     \item sentence segmentation
%     \item tokenization
%     \item POS tagging
%     \item lemmatization
    
% \end{itemize}
% \end{frame}

% \begin{frame}
% \frametitle{Ensuring quality}

% \begin{itemize}
%     \item reproducibility
%     \item collaborations
%     \item availability for all
    
% \end{itemize}

% \end{frame}

% \section{Example}
% \begin{frame}
% \frametitle{}


% \end{frame}

% \begin{frame}
% \frametitle{}


% \end{frame}


% \begin{frame}
% \frametitle{}


% \end{frame}


% \begin{frame}
% \frametitle{}


% \end{frame}


% \begin{frame}
% \frametitle{}


% \end{frame}


% \begin{frame}
% \frametitle{}


% \end{frame}




 
\end{document}